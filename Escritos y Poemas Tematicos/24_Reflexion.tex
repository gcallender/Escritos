%\currentpdfbookmark{Sensaciones}{Sensaciones}
\refstepcounter{reflexion}
\section*{Reflexión}
\label{Reflexion}
\addcontentsline{toc}{section}{Reflexión}

% Contenido
\vspace{1em}
\begin{center}
Hace poco una frase escuché,\\ 
que toda persona está sola al nacer,\\ 
vamos a pensar sobre su veracidad,\\ 
a analizar si la soledad es real.

\vspace{1em} 
Cuando se nace siempre alguien hay,\\ 
ya que todos una madre han de tener,\\ 
quien compañía ha de entregar,\\ 
aunque sean segundos o una eternidad.

\vspace{1em} 
La soledad no existe de forma literal,\\ 
solo de forma emocional,\\ 
todo aquel que solo desea estar,\\ 
o quien solo ha de quedar,\\ 
es una opción viable a tomar.

\vspace{1em} 
Todo ser puede acompañar,\\ 
neutralizando así a la soledad,\\ 
ahora todo depende de quién enfrente esté,\\ 
si acepta o no dicha oportunidad.

\vspace{1em} 
Todo en el universo se desarrolla en comunidad,\\ 
todos somos parte de un gran plan,\\ 
aprender a mirar hacia todos lados,\\ 
y poner en práctica lo que es el respetar.

\vspace{1em} 
Se invita a aquel que cree que solo está,\\ 
a una noche el cielo observar,\\ 
años luz separan una estrella de otra,\\ 
pero aún así su luz no deja de iluminar,\\ 
con la cual cada una es capaz de comunicar,\\ 
que la distancia no es sinónimo de soledad.

\vspace{1em} 
Continuando con la frase anterior,\\ 
se dice que una persona al morir,\\ 
enfrenta sola dicha situación,\\ 
vamos a dar otra vuelta,\\ 
y pensar si esas palabras tienen razón.

\vspace{1em} 
Incluso la muerte genera unión,\\ 
la sociedad depende de dicha comunión,\\ 
es necesario despedir a alguien,\\ 
para comprender un tipo de amor.

\vspace{1em} 
Varios afrontan el morir en soledad,\\ 
y acá es donde falla la comunidad,\\ 
creer que uno vive en un círculo de amistad,\\ 
es una forma de aislamiento y de segregar.

\vspace{1em} 
Esta vez existe una paridad,\\ 
el paso a otro mundo se puede solo realizar,\\ 
pero este hecho a su vez se puede dar,\\ 
con la compañia que cada uno quiera adoptar.

\vspace{1em} 
Para culminar un relato más,\\ 
queda algo que es necesario analizar,\\ 
que hago yo en este mundo,\\ 
para incrementar el aislamiento y la soledad.

\end{center}




