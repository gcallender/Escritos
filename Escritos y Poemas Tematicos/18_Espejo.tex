%\currentpdfbookmark{Sensaciones}{Sensaciones}
\refstepcounter{espejo}
\section*{Espejo}
\label{Espejo}
\addcontentsline{toc}{section}{Espejo}

% Contenido
\vspace{1em}
\begin{center}
Un sentimiento está en extinción,\\ 
a la distancia una flor se marchita,\\ 
de pronto ocurre un resplandor,\\ 
al olvido ha pasado su corta vida.

\vspace{1em} 
Interesante el mundo vivo,\\ 
en un momento se tiene felicidad,\\ 
para luego conocer la antítesis,\\ 
que algunos llaman soledad.

\vspace{1em} 
Un pequeño te da una mano,\\ 
la alegría surge en profundidad,\\ 
una voz de mando se logra escuchar,\\ 
``cuidado, no te vayas a contagiar".

\vspace{1em} 
Si un árbol es derribado,\\ 
muchos sueños quedan sin hogar,\\ 
la esperanza se ha esfumado,\\ 
y algunos se han de alegrar.

\vspace{1em} 
Aves que cesan su cantar,\\ 
peces que dejan de volar,\\ 
producen una pequeña herida,\\ 
en quien al mundo vigila,\\ 
esperando en eterna vigilia,\\ 
que una lágrima sea de corazón,\\ 
por quienes se van sin razón.

\vspace{1em} 
Un juicio ha de haber,\\ 
el juzgado solo de un lado está,\\ 
la situación tal vez cambiaría,\\ 
si de cabeza se le ocurriera pensar.

\vspace{1em} 
Dicen que el universo se mueve,\\ 
que crece constantemente,\\ 
mientras tanto quienes lo creen,\\ 
observan desde una vitrina,\\ 
como el tiempo se lleva la vida,\\ 
de quien no es capaz de ver,\\ 
que la flor aún se marchita.

\vspace{1em} 
La luz está en quien,\\ 
aunque sin tener motivo,\\ 
es capaz de comprender,\\ 
que aquel árbol sigue vivo.

\vspace{1em} 
Sus hojas ya se han caído,\\ 
un ciclo ha de continuar,\\ 
el reloj aún contiene arena,\\ 
este tiempo lo han detenido,\\ 
por quien solo un deseo lo llena,\\ 
que las hojas vuelvan a brillar,\\ 
para que puedan observar,\\ 
como este pequeño heredero,\\ 
es capaz de devolver y ramificar,\\ 
el sentimiento que lo ha de iluminar.


\end{center}




