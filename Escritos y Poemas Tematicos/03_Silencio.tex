%\currentpdfbookmark{Silencio}{Silencio}
\refstepcounter{silencio}
\section*{Silencio}
\label{Silencio}
\addcontentsline{toc}{section}{Silencio}
%\markboth{Silencio}{Silencio}
%\pdfbookmark[0]{Silencio}{Silencio}

% Contenido
\vspace{1em}
\begin{center}
Silencio, ausencia de sonido que abre los sentidos,\\
vibración no presente que se conecta con la mente,\\
mente que oye solamente lo que quiere oír,\\
o tal vez siente lo que debe sentir.

\vspace{1em}
El silencio es un concepto físico,\\ 
que conlleva múltiples consecuencias\\
dependiendo de quien lo perciba o quien lo genere,\\
lo cual ocurre en su debido tiempo;\\
qué sería de las comunicaciones sin el silencio,\\
qué sería de la sociedad sin el silencio,\\
qué sería de la vida sin el silencio.

\vspace{1em}
La naturaleza no conoce el silencio,\\
naturaleza omnipresente que percibe todo,\\
en algún lugar ha de haber siempre algún sonido,\\
alguna vibración sonora que avisa que todo está vivo,\\
por ende, ¿el silencio es muerte?\\
Bajo ciertas circunstancias si lo es,\\
pero solo desde un punto de vista de quién lo quiera ver así,\\
aún la muerte no existe como tal,\\
dentro de la misma muerte existe vida,\\
así como en la vida hay muerte.\\
Con el silencio ocurre algo similar,\\
sin sonido no hay silencio y sin silencio no hay sonido,\\
para quien solo escucha o para quien solo ve lo que quiere ver.

\vspace{1em}
Las percepciones son subjetivas,\\
subjetividad que da espacio a interpretaciones propias,\\
cuando se produce algún ruido,\\
en ese mismo instante se está produciendo\\ 
un silencio en otro lugar,\\
una pausa en el tiempo,\\
se ha interrumpido un flujo de vibraciones\\ 
que deseaban llegar a cierto destino,\\
destino que nunca se sabrá,\\
destino que jamás se enterará que deseaban llegar hasta él.

\vspace{1em}
Un silencio es una pausa en la vida,\\
vida que fluye al ritmo de vibraciones,\\
música, ruidos, sonidos diversos\\
que alteran los sentidos\\ 
de los variados seres que comparten este mundo.


\end{center}




